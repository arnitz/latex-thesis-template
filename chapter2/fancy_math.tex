% **************************************************************************************************
% **************************************************************************************************
\newsection{Higher Mathematics}{fancy:math}

Naturally, there are also several commands that make life easier when dealing with equations. One of the central ideas is to be able to change the general style of something, for example vector/matrix highlighting ($\vm{\phi}$ vs. $\phi$), just by modifying the template command.

\nxtpar\noindent
Here are a few examples. Note that these equations, \req{fancy:math:1} through \req{fancy:math:3b}, do not necessarily make sense in a mathematical sense.
\begin{equation}
  \var{a + b} \isreq \var{a} + \var{b} + 2 \cov{a,b}
  \label{eq:fancy:math:1}
\end{equation}
\begin{equation}\begin{split}
  \vm{H}
  &\isdef \exp{\E{\vm{h}^T \vm{h}}} - \ln{\vm{h}^T \vm{h}} + \log{\vm{h}^T \vm{h}} - \frac{\ld{\vm{h}^T \vm{h}}}{\logb{3}{\vm{h}^T \vm{h}}} \\
  &= \mtx{ccc}{h1 & h2 & \dots \\ 0 & h1 & \dots \\ \vdots & \vdots & \ddots}
  \label{eq:fancy:math:2}
\end{split}\end{equation}
\begin{align}
  \E{ a b\conj c d\conj} &= \E{a b\conj} \cdot \E{c d\conj} + \E{a d\conj} \cdot \E{c b\conj} \label{eq:fancy:math:3a}\\
   \E{a b\conj} \cdot \E{c d\conj} &\neq  \E{a d\conj} \cdot \E{c b\conj} - \E{ a b\conj c d\conj}\label{eq:fancy:math:3b}
\end{align}